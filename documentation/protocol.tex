\documentclass[11pt, oneside]{article}  
\usepackage[parfill]{parskip} 
\usepackage{graphicx}
\usepackage{amssymb}
\usepackage[iso,english]{isodate}    
\usepackage{fontspec}
\usepackage{tikz}
\usetikzlibrary{shapes,snakes}
\usepackage{color}

\usepackage[paperwidth=16cm,paperheight=25cm,margin=0cm]{geometry}
\usepackage[a4,center,noinfo]{crop}
\usepackage{fancyhdr}

\usepackage[a4,center,noinfo]{crop}
\definecolor{airforceblue}{rgb}{0.36, 0.54, 0.66}
\definecolor{asparagus}{rgb}{0.53, 0.66, 0.42}
\definecolor{bisque}{rgb}{1.0, 0.89, 0.77}
\definecolor{bittersweet}{rgb}{1.0, 0.44, 0.37}
\definecolor{brickred}{rgb}{0.8, 0.25, 0.33}
\definecolor{buff}{rgb}{0.94, 0.86, 0.51}
\definecolor{celadon}{rgb}{0.67, 0.88, 0.69}
\definecolor{corn}{rgb}{0.98, 0.93, 0.36}
\definecolor{darkpastelblue}{rgb}{0.47, 0.62, 0.8}
\definecolor{salmon}{rgb}{1.0, 0.55, 0.41}

\definecolor{burlywood}{rgb}{0.87, 0.72, 0.53}
\definecolor{darkpastelpurple}{rgb}{0.59, 0.44, 0.84}
\definecolor{verylightgray}{rgb}{0.9, 0.9, 0.9}

\setmainfont[Ligatures=TeX,ItalicFont={Elena Basic Regular Italic}]{Elena Basic}

\title{UHPPOTE-UTO0311-L0x - PROTOCOL}
\author{}
\date{\today}

\pagestyle{fancy} 
\lhead{\footnotesize UHPPOTE UTO0311 L0X}
\chead{\footnotesize PROTOCOL DESCRIPTION}

\lfoot{\footnotesize UHPPOTE-UT0311-L0X Rev. 0}
\cfoot{\footnotesize UNCLASSIFIED}
\rfoot{\footnotesize \thepage}

\renewcommand{\footrulewidth}{0.4pt}

% -- MACRO'S

% -- TITLE PAGE

\begin{document}

\thispagestyle{empty}

\begin{tikzpicture}
\draw [fill=white]   (0,0) rectangle(16,24.9);

%\node[anchor=north]  (logo) at (8,20) 
%  { \includegraphics[width=8.5cm]{twyst.png}
%  };

\node[anchor=north]  (title)  at (8,12.5) 
  { \fontsize{16}{16}
    \selectfont
    \textbf{UHPPOTE UTX031 L0X}
 };

\node[anchor=north]  (subtitle)  at (8,11.25) 
  { \fontsize{13}{16} 
    \selectfont
    \textit{Protocol Description}
 };
  
\end{tikzpicture}

% -- CONTENT

\clearpage

\setcounter{page}{1}
\renewcommand{\arraystretch}{1.75}
   
\begingroup\centering 
\fontsize{16}{20}
\selectfont
\textbf{UHPPOTE UTO0311 L0X}
\vspace{1cm}

\fontsize{13}{16}
\selectfont
\textit{Protocol Description}
\vspace{0.25cm}

\hrulefill
\endgroup

% -- OVERVIEW

\section*{Overview}

The UHPPOTE-UTO031-L0X protocol is a UDP packet protocol that provides the functionality required to:

\begin{itemize}  
\item identify the UHPPOTE-UTO031-L0X controllers present on a TCP/IP network
\item configure the controllers present on a TCP/IP network
\item listen for access events from controllers present on a TCP/IP network
\end{itemize}

\textbf{
}

A request is 64 byte UDP packet with the following general format:

{ \begin{tikzpicture}[]
  \draw [] (0,0)    rectangle (16,1); 
  \draw [] (0,0)    rectangle (1.2,1); 
  \draw [] (1.2,0)    rectangle (2.7,1); 
  \draw [] (2.7,0) rectangle (16,1); 
  \draw (0.6,0.5) node [] { \large 0x17 }; 
  \draw (1.95,0.5) node [] { \large CODE }; 
  \draw (2.75,0.5) node [anchor=west] { \large PARAMETERS\ldots }; 
 \end{tikzpicture}
}

A response is 64 byte UDP packet with the following general format:

{ \begin{tikzpicture}[]
  \draw [] (0,0)    rectangle (16,1); 
  \draw [] (0,0)    rectangle (1.2,1); 
  \draw [] (1.2,0)    rectangle (2.7,1); 
  \draw [] (2.7,0) rectangle (16,1); 
  \draw (0.6,0.5) node [] { \large 0x17 }; 
  \draw (1.95,0.5) node [] { \large CODE }; 
  \draw (2.75,0.5) node [anchor=west] { \large DATA\ldots }; 
 \end{tikzpicture}
}

Message elements

{ \begin{tabular}{ | l | l | l | }
   \hline
   & TYPE & ENCODING \\ 
   \hline
   Device ID & uint32 & binary \\ 
   \hline
   Card Number & uint32 & binary \\ 
   \hline
   Date & YYMMDD & BCD \\ 
   \hline
   Time & HHmm & BCD \\ 
   \hline
   IPv4  & N.N.N.N & binary[4] \\ 
   \hline
   MAC  & X.X.X.X.X & binary[6] \\ 
   \hline
   \end{tabular}
   \vspace{0.5cm}
}


{ \begin{tabular}{ | l | c | l | }
   \hline
   COMMAND & CODE & DESCRIPTION \\ 
   \hline
   find-devices & 0x94 &  Broadcast command to identify all devices on a subnet \\ 
   \hline
   get-status &  &  \\ 
   \hline
   get-time &  &  \\ 
   \hline
   get-cards &  &  Get the number of cards stored in the internal access list \\ 
   \hline
   get-swipes &  &  \\
   \hline
   set-time &  &  \\
   \hline
   set-ip-address &  &  \\
   \hline
   grant &  &  \\
   \hline
   revoke &  &  \\ 
   \hline
   open-door &  &  \\ 
   \hline
   \end{tabular}
   \vspace{0.5cm}
}

% -- find-devices

\section*{FIND DEVICES}

UDP broadcast command to identify all UHPPOTE-UTO031-L0X controllers present on the local TCP/IP network.

REQUEST:

{ \begin{tabular}{ | l | l | l | l | }
   \hline
   OFFSET & LENGTH & FIELD & VALUE \\ 
   \hline
   0 & 1 & SOM & 0x17 \\ 
   \hline
   1 & 1 & CODE & 0x94 \\ 
   \hline
   2 & 1 & PAD & 0x00 \\ 
   \hline
   3 & 1 & PAD & 0x00 \\ 
   \hline
   4 & 60 & PADDING & 0x00 0x00 \ldots \\ 
   \hline
   \end{tabular}
   \vspace{0.5cm}
}

RESPONSE:

{ \begin{tabular}{ | l | l | l | l | }
   \hline
   OFFSET & LENGTH & FIELD & VALUE \\ 
   \hline
   0 & 1 & SOM & 0x17 \\ 
   \hline
   1 & 1 & CODE & 0x94 \\ 
   \hline
   2 & 1 & PAD & 0x00 \\ 
   \hline
   3 & 1 & PAD & 0x00 \\ 
   \hline
   4 & 4 & SERIAL NUMBER & uint32 \\ 
   \hline
   8 & 4 & IP ADDRESS & IPv4 \\ 
   \hline
   12 & 4 & SUBNET MASK & IPv4 \\ 
   \hline
   16 & 4 & GATEWAY ADDRESS & IPv4 \\ 
   \hline
   20 & 6 & MAC ADDRESS & MAC \\ 
   \hline
   26 & 2 & VERSION & VERSION \\ 
   \hline
   28 & 6 & DATE & DATE \\ 
   \hline
   34 & 30 & PADDING & 0x00 0x00 \ldots \\ 
   \hline
   \end{tabular}
   \vspace{0.5cm}
}

Example:

\texttt{{\color{blue}\textbf{17}} 
{\color{red}\textbf{94}} 
00 00 00 00 00 00  00 00 00 00 00 00 00 00}\newline
\texttt{00 00 00 00 00 00 00 00  00 00 00 00 00 00 00 00}\newline
\texttt{00 00 00 00 00 00 00 00  00 00 00 00 00 00 00 00}\newline
\texttt{00 00 00 00 00 00 00 00  00 00 00 00 00 00 00 00}\newline

\texttt{{\color{blue}\textbf{17}} 
{\color{red}\textbf{94}} 
00 00 2d 55 39 19  c0 a8 01 7d ff ff ff 00}\newline
\texttt{ff ff ff 00 00 66 19 39  55 2d 08 92 20 18 08 16}\newline
\texttt{00 00 00 00 00 00 00 00  00 00 00 00 00 00 00 00}\newline
\texttt{00 00 00 00 00 00 00 00  00 00 00 00 00 00 00 00}\newline


%{ \begin{tikzpicture}[]
%  \draw [] (0,0)    rectangle (16,1); 
% \draw [] (0,0)    rectangle (1,1); 
%  \draw [] (1,0)    rectangle (2,1); 
%  \draw [] (2,0)    rectangle (3,1); 
%  \draw [] (2,0)    rectangle (4,1); 
%  \draw [] (2,0)    rectangle (16,1); 
%  \draw (0.5,0.5) node [] { \large 17 }; 
%  \draw (1.5,0.5) node [] { \large 94 }; 
%  \draw (2.5,0.5) node [] { \large 00 }; 
%  \draw (3.5,0.5) node [] { \large 00 }; 
% \end{tikzpicture}
%}

%{ \begin{tikzpicture}[]
%  \draw [] (0,0)    rectangle (16,1); 
 % \draw [] (0,0)    rectangle (1,1); 
%  \draw [] (1,0)    rectangle (2,1); 
%  \draw [] (2,0)    rectangle (3,1); 
%  \draw [] (2,0)    rectangle (4,1); 
%  \draw [] (4,0)    rectangle (5.8,1); 
%  \draw [] (5.8,0)    rectangle (7.7,1); 
%  \draw [] (7.7,0)    rectangle (9.15,1); 
%  \draw [] (9.15,0)    rectangle (11.1,1); 
%  \draw [] (11.1,0)    rectangle (12.4,1); 
%  \draw [] (12.4,0)    rectangle (14.1,1); 
%  \draw (0.5,0.5) node [] { \large 17 }; 
%  \draw (1.5,0.5) node [] { \large 94 }; 
%  \draw (2.5,0.5) node [] { \large 00 }; 
%  \draw (3.5,0.5) node [] { \large 00 }; 
%  \draw (4.1,0.5) node [anchor=west] { \small SERIAL \# }; 
%  \draw (5.9,0.5) node [anchor=west] { \small ADDRESS }; 
%  \draw (7.8,0.5) node [anchor=west] { \small MASK }; 
%  \draw (9.2,0.5) node [anchor=west] { \small GATEWAY }; 
%  \draw (11.2,0.5) node [anchor=west] { \small MAC }; 
%  \draw (12.4,0.5) node [anchor=west] { \small VERSION }; 
%  \draw (14.25,0.5) node [anchor=west] { \small DATE }; 
% \end{tikzpicture}
%}

\end{document}  